\documentclass{seminar}
\usepackage{fancybox}
 \usepackage{pstricks} 
\usepackage{graphicx}
\usepackage{fancyhdr} % Headers and footers personalization using the `fancyhdr' package
 \fancyhf{} % Clear all fields 

\renewcommand{\printlandscape}{\special{landscape}}
\def\FileAuthor{\textcolor{red}{\textsf{pre\_lmgc}}}
  \def\FileTitle{\textbf{LMGC team}}
  \def\FileDate{\textcolor{blue}{Octobre 2006}}
\def\FileSubject{\textcolor{blue}{Tout Python}}
  
 % \hypersetup{
 %   pdftitle      = \FileTitle{},
 %   pdfsubject    = \FileSubject{},
 %   pdfkeywords   = {LaTeX, PDF},
 %   pdfauthor     = \FileAuthor{},
 %   pdfcreator    = {\LaTeX\ with package \flqq hyperref\frqq},
 % }
  
  \slidesmag{2}     % Set magnification of slide
  \slideframe{none} % No default frame
  
  % General size parameters
  \renewcommand{\slideparindent}{5mm}
  \raggedslides[0mm]
  %\setlength{\slidewidth}{210truemm}
  %\setlength{\slideheight}{297truemm}
  %\def\SeminarPaperWidth{210truemm}
  %\def\SeminarPaperHeight{297truemm}
  
  %\renewcommand{\slidetopmargin}{14.5mm}
  %\renewcommand{\slidebottommargin}{13.5mm}
  %\renewcommand{\slideleftmargin}{2mm}
  %\renewcommand{\sliderightmargin}{2mm}
  
  % To adjust the frame length to the header and footer ones
  %\autoslidemarginstrue
  
  % We suppress the header and footer `fancyhdr' rules
  \fancyhf{} % Clear all fields
  \renewcommand{\headrule}{}
  \renewcommand{\footrule}{}
  
  % Headers and footers personalization using the `fancyhdr' package
  \fancyhead[L]{\small \FileTitle{}}
  \fancyhead[R]{\small \FileAuthor{}}
  \fancyfoot[L]{\small \FileDate{}}
  \fancyfoot[C]{\small \FileSubject{}}
  \fancyfoot[R]{\small Page \theslide}



\begin{document}
\pagestyle{fancy}

%-------------------------------------------------------------
%	DIAPOSITIVE 1
%-------------------------------------------------------------
\begin{slide}
\slideframe{none}
\begin{center}
\textbf{Pre-Processeur LMGC90}  
\end{center}
\end{slide}

%-------------------------------------------------------------
%	DIAPOSITIVE 2
%-------------------------------------------------------------
\begin{slide}{\textbf{\textcolor{red}{Buts}}} Pr\'eprocesseur LMGC90
\slideframe{none}
\begin{description}
\item \textsf{\textcolor{blue}{MultiCorps}}
\item \textsf{\textcolor{blue}{MultiPhysiques}} 
\item \textsf{\textcolor{blue}{Programmation Orient\'e Objet}}
\end{description}
\end{slide}

%-------------------------------------------------------------
%	DIAPOSITIVE 3
%-------------------------------------------------------------
\begin{slide}{\textbf{\textcolor{red}{Installation}}} 
\slideframe{none}
\begin{description}
\item \textsf{\textcolor{blue}{Modules annexes}}(hors distribution
	    python) \texttt{numarray}  \texttt{vtk}

\item \textsf{\textcolor{blue}{repertoire d'installation}} = \texttt{PRE\_LMGC} 
\item \textsf{\textcolor{blue}{sous r\'epertoire}} un fichier \texttt{pre\_lmgc.py}
	    \begin{enumerate}
	     \item \texttt{domain} contient l'ensemble des donnees pour
		   construire des \texttt{BODIES}
             \item \texttt{interaction} pas grand chose pour le moment
             \item \texttt{output\_files} place des fichiers de sorties
		   par defaut.
             \item \texttt{files} contient les fichiers permettant d
		   ecrire les fichiers de sorties
             \item \texttt{material} contient des fichiers permettant de
		   gerer les materiaux
             \item \texttt{common} contient des donnees pour tous.
	    \end{enumerate}
\item \texttt{\textcolor{blue}{PYTHONPATH}} pour utiliser pre\_lmgc de n
	    importe o\`u il faut modifier 
        la variable d environnement \texttt{PYTHONPATH=.:PRE\_LMGC:PRE\_LMGC/material/:PRE\_LMGC/common/}
\end{description}
\end{slide}


%-------------------------------------------------------------
%	DIAPOSITIVE 4
%-------------------------------------------------------------
\begin{slide}{\textbf{\textcolor{red}{Informations}}} session int\'eractive
\slideframe{none}
\begin{description}
\item \texttt{\textcolor{blue}{python}} lance python en session
	    int\'eractive
\item \texttt{\textcolor{blue}{importation du module pre\_lmgc}}
	    \texttt{from pre\_lmgc import *}
\item \texttt{\textcolor{blue}{help(...)}} pour trouver une information
	    sur une classe,
	    une methode de classe,...etc. Exemple : \texttt{help(mesh)}
\end{description}
\begin{verbatim}
 classe permettant de definir les parties maillees a partir d un fichier de maillage
        Methodes :
          - __init__
          - lecture
          - defineModele
          - defineMateriau
          - defineContacteurs
          - imposeDrivenDof
          - imposeInitValue
          - rotateBody
          - moveBody
          - addNodes
          - addGroups
          - addElements

\end{verbatim}
\end{slide}
%-------------------------------------------------------------
%	DIAPOSITIVE 4
%-------------------------------------------------------------

\begin{slide}{\textbf{\textcolor{red}{Maillage}}} d\'efinition dans \texttt{domain/MAILx}
\slideframe{none}
\begin{description}
\item \texttt{\textcolor{blue}{lecture d un fichier }}
	    \texttt{p=mesh(fichierMaillage='../epr.msh')} \\
	    \texttt{print p} donne des informations
	    \texttt{On pense \`a orienter les contours dans le sens trigo}.
\item \texttt{\textcolor{blue}{Instance de mesh}} contient des
	    ensembles:
	   \begin{enumerate}
	    \item d'\'el\'ements contenue sous \texttt{p.elements}
            \item de noeuds contenue sous \texttt{p.nodes}
            \item de groups contenue sous \texttt{p.groups}
	    \end{enumerate}
\item \texttt{\textcolor{blue}{iterateur}} chacun des ensembles est un
	    iterateur \texttt{common/iterateur.py} avec des m\'ethodes
	    \\
	    Pour avoir  l ensemble des groupes : \texttt{print
	    p.groups.liste()} \\
	    On remarquera qu'il y a un groupe par entit\'e physique mais aussi
	    par type d'\'el\'ements et que le groupe 'tous' est d\'efini.
\end{description}
\end{slide}

%-------------------------------------------------------------
%	DIAPOSITIVE 5
%-------------------------------------------------------------
\begin{slide}{\textbf{\textcolor{red}{Maillage}}} Les groupes
\slideframe{none}
\begin{description}
\item On peut trouver un ensemble d'une entit\'e (\texttt{nodes} ou
	    \texttt{elements}) \`a l'aide de la m\'ethode \texttt{find}
d'une instance de \texttt{mesh}
	    exemple : \texttt{p.find(type='nodes',group='6')}
\item \texttt{\textcolor{blue}{Modification d'un nom}}
	    \texttt{p.groups.modifieNom('6','bordGauche')}
\item \texttt{\textcolor{blue}{Modification de plusieurs  noms}}
	    \texttt{p.groups.modifieNoms({'6': 'bordGauche','9':'bordDroit'})}
\end{description}
\end{slide}

%-------------------------------------------------------------
%	DIAPOSITIVE 6
%-------------------------------------------------------------
\begin{slide}{\textbf{\textcolor{red}{Maillage}}} Les mod\`eles
\slideframe{none}

\begin{description}
\item \texttt{\textcolor{blue}{class model}} pour d\'efinir un
	    mod\`ele. Lorsqu'on definit un modele, on d\'efinit
	    automatiquement un type de d.d.l associ\'e.
\item \texttt{\textcolor{blue}{exemple}} : \texttt{m =model('T3DNL',element='T3xxx',type='THERx',capaStorage='lump\_')} \\
	    \texttt{print m} ; pour des informations. 
\item  \texttt{\textcolor{blue}{mesh}} : association d un mod\`ele est
	    d'une
	    partie. \texttt{p.defineModele(group='T3xxx',modele=m)} le
	    'model' ne s'appliquera qu'aux \'el\'ements g\'eom\'etriques
	    compatibles avec ceux du 'model'.
	    \textit{defineModele(modele='T3DNx') est aussi valable mais
	    il n y a plus la possibilit\'e d'imposer des d.d.l.}
\item \texttt{\textcolor{blue}{conteneur}} \texttt{ms=models();ms.addModele(m)}
\end{description}
\end{slide}

%-------------------------------------------------------------
%	DIAPOSITIVE 7
%-------------------------------------------------------------
\begin{slide}{\textbf{\textcolor{red}{Maillage}}} Les mat\'eriaux
\slideframe{none}

\begin{description}
\item \texttt{\textcolor{blue}{class material}} pour definir un
	    mat\'eriau
\item \texttt{\textcolor{blue}{exemple}} : \texttt{mat =material(nom='acier',type='ELAS',elas\_modele='1',young=200000,nu=0.3)} \\
	    \texttt{print mat} ;
\begin{verbatim}
Materiau:acier
        Comportement de type    :ELAS
        Proprietes definies :
                               young    :       2000000.0
                                  nu    :       0.3
                         elas_modele    :       elas \Neo Hookeen \hyper
\end{verbatim}
\end{description}
\textbf{Tout ne doit pas encore \^etre d\'efini}
\end{slide}
%-------------------------------------------------------------
%	DIAPOSITIVE 8
%-------------------------------------------------------------
\begin{slide}{\textbf{\textcolor{red}{Le conteneur de parts}}} \texttt{domain/parts.py}
\slideframe{none}

\begin{description}
\item \texttt{\textcolor{blue}{class parts}} cr\'eation d'une instance
	    \texttt{ps=parts()}
\item \texttt{\textcolor{blue}{Ajout d'une part}}
	    \texttt{p=mesh(numero='1',fichierMaillage='tt.msh');ps.addPart(p)}
\item \texttt{\textcolor{blue}{Acc\`es \`a une part}} \texttt{print
	    ps['1']} pour imprimer les infos sur une partie.
\item \texttt{\textcolor{blue}{Relecture d un 'BODIES.DAT'}}
	    ps=parts\_lmgc(chemin='') !A REVOIR
\end{description}
\end{slide}
%-------------------------------------------------------------
%	DIAPOSITIVE 9
%-------------------------------------------------------------
\begin{slide}{\textbf{\textcolor{red}{Un exemple complet}}} sans imposition de dof 
\slideframe{none}
\begin{verbatim}
from pre_lmgc import * # importation du module pour le pre-processeur

p=mesh(numero='1',fichierMaillage='epr.msh') # lecture du maillage
                                             # 'epr.msh' et reperage par le numero 1

p.defineModele(modele='T3DNL')		     # definition d un nom pour le modele
p.defineMateriau(materiau='acier)            # definition du nom du materiau
ps=parts()

ps.addPart(p)

writeBodies(ps)				     # ecriture du fichier
                                             # 'BODIES.DAT' dans le repertoire pointe par la variable
                                             # OUTPUT_DIR de common/variables.py

\end{verbatim}
\end{slide}
%-------------------------------------------------------------
%	DIAPOSITIVE 10
%-------------------------------------------------------------
\begin{slide}{\textbf{\textcolor{red}{Les d.d.l}}}  
\slideframe{none}
\begin{description}
\item \texttt{\textcolor{blue}{D\'efinition}} \`a la cr\'eation du mod\`ele
\item \texttt{\textcolor{blue}{Affectation}} \`a l'affectation d'un
	    mod\`ele sur un groupe.
\item \texttt{\textcolor{blue}{Imposition}} sur un groupe pour un
	    mod\`ele et suivant une composante d'une valeur initiale ou
	    impos\'ee sous la forme d'une fonction \texttt{evolution}
\item \texttt{\textcolor{blue}{Stockage}} en chaque noeud dans un
	    conteneur \texttt{ddls} dont les cl\'e sont celles du mod\`eles. 
\item \texttt{\textcolor{blue}{Exemple}} p est une instance de mesh

\begin{verbatim}
...
>>>m=model(nom='M3DNL',type='MECAx',element='T3xxx')
>>>p.defineModele(group='T3xxx',modele=m)
>>>p.imposeDrivenDof(group='10',modele='MECAx',composante=1,ct=10.)
....
>>>writeDrvDof(ps)
>>>print p.nodes['1'].ddls['MECAx']
        ddl de type     :       vecteur  repere         :       global
                        composante 1 :  0.0      impose         :        True 
                        composante 2 :  0.0      impose         :        False
                        composante 3 :  0.0      impose         :        False 
\end{verbatim}	

\end{description}
\end{slide}
%-------------------------------------------------------------
%	DIAPOSITIVE 10
%-------------------------------------------------------------
\begin{slide}{\textbf{\textcolor{red}{Les Contacteurs}}}  
\slideframe{none}
\begin{description}
\item \texttt{\textcolor{blue}{Cr\'eation}}
	    \texttt{p.defineContacteurs(...)}
	    cr\'ee un conteneur p.contacteurs d\'erivant d'\texttt{iterateur}�
\item \texttt{\textcolor{blue}{Caract\'eristiques}} voir d\'efinition
	    des candidats et antagonistes.
\item \texttt{\textcolor{blue}{Exemple}} 
\begin{verbatim}
...
>>>p.defineContacteurs(group='Line',type='CLxxx',apab=0.5)
...
>>>writeBodies(ps,chemin='')
\end{verbatim}
V\'erifier le \texttt{BODIES.DAT}. \\
Certains passage sont optionnel suivant le type de contacteurs ici
	    \texttt{apab}. Si on refait une d\'efinition de contacteurs
	    alors on va rajouter des contacteurs sur de m\^eme
	    \'el\'ements 'lignes', ...
\end{description}
\end{slide}
\end{document}
